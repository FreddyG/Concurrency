\documentclass[a4paper]{article}

\usepackage[english]{babel}
\usepackage{amsmath}
\usepackage{graphicx}

\author{Maarten de Jonge}
\date{\today}
\title{Concurrency and Parallel Programming \\
\large{Assignment 3}}

\begin{document}
\maketitle

\section*{Wave Equation}
\begin{table}[htbp]
    \centering
    \begin{tabular}{|l|l|l|l|l|}
        \hline
        i\_max    & t\_max   & num\_threads & time pthreads   & time OMP \\
        \hline
        1000     & 1000000 & 1           & 42.4294 & 2.7425  \\
        1000     & 1000000 & 2           & 31.5108 & 42.3953 \\
        1000     & 1000000 & 4           & 80.9137 & 20.493  \\
        1000     & 1000000 & 8           & 167.284 & 23.8034 \\
        1000     & 1000000 & 16          & 350.395 & 21.2966 \\
        10000    & 500000  & 1           & 69.3228 & 12.4777 \\
        10000    & 500000  & 2           & 44.066  & 85.0008 \\
        10000    & 500000  & 4           & 41.1647 & 47.4733 \\
        10000    & 500000  & 8           & 85.3845 & 36.9657 \\
        10000    & 500000  & 16          & 183.268 & 33.3651 \\
        1000000  & 5000    & 1           & 52.108  & 15.6997 \\
        1000000  & 5000    & 2           & 25.6232 & 83.3668 \\
        1000000  & 5000    & 4           & 13.1806 & 42.1536 \\
        1000000  & 5000    & 8           & 7.08639 & 27.3887 \\
        1000000  & 5000    & 16          & 7.06076 & 29.6989 \\
        10000000 & 500     & 1           & 51.4277 & 15.1485 \\
        10000000 & 500     & 2           & 25.8168 & 83.1574 \\
        10000000 & 500     & 4           & 13.0537 & 41.439  \\
        10000000 & 500     & 8           & 7.21982 & 27.4845 \\
        10000000 & 500     & 16          & 7.16489 & 36.9947 \\
        \hline
    \end{tabular}
    \caption{The raw test data, where ``i\_max'' is the number of simulated points
    on the wave and ``t\_max'' is the amount of iterations.}
    \label{table:results}
\end{table}

Table \ref{table:results} shows the results both using pthreads and using
OpenMP. A few things are noticable here:
\begin{itemize}
    \item The situations with 1 thread are significantly faster with OpenMP.
          This is posibly because the pthread version of the code spawns a new thread
          for each time iteration, even though that's unnecessary when only using a
          single thread. OpenMP might handle this more graciously.
    \item In general, it seems OpenMP handles the "bad" cases (the ones where
          there's not enough workload per thread to be worthwhile) better than the
          threaded code, with a smaller performance loss.
    \item When the amount of simulated points becomes high enough, the threaded
          solution is faster than the OpenMP one every time.
    \item With the OpenMP code, the optimal number of threads is 1 for every set
          of parameters; using more threads actually slows it down. This might indicate
          that I did not use OpenMP optimally.
\end{itemize}


\end{document}
