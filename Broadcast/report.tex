\documentclass[a4paper]{article}

\usepackage[english]{babel}
\usepackage{amsmath}
\usepackage{graphicx}

\author{Maarten de Jonge}
\date{\today}
\title{Concurrency and Parallel Programming \\
\large{Assignment 2}}

\begin{document}
\maketitle

\section*{Wave Equation}
Unfortunately, due to a lack of time on my part, this part of the assignment is
not finished; the code is messy and doesn't run correctly yet. The attempted
implementation is similar to last week's assignment, except that the process
ranked 0 will send slices of the array to each process. This means that the
program would require at least two MPI processes, otherwise there would be no
worker processes available.

\section*{Collective Communication}
The broadcast function is implemented with the assumption that it's running on a
ring-shaped network, where each process is connected to the processes with a
rank of 1 higher/lower. The process opposite the root is determined as having
the rank $r_{root} + (N / 2)$, where $r_{root}$ is the rank of the root and $N$
is the total number of processes. The root sends its message to both its
neighbours, which each pass it on until the messages converge at the process
opposite of the root, which can receive the message from either side. A simple
'main' function is included to demonstrate that it works correctly.

\end{document}
