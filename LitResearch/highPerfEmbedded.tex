\documentclass[a4paper]{article}

\usepackage[english]{babel}
\usepackage{amsmath}
\usepackage{graphicx}

\author{Maarten de Jonge}
\date{\today}
\title{On the convergence of high-performance and embedded computing}

\begin{document}
\maketitle

\section{Introduction}
Traditionally, high-performance and embedded computing have been entirely
seperate entities; back in the 1990s, phones could do little more than make
calls or send text messages, and nobody would consider running large data
analysis on their microwave. Recent advancements in the fields of multi-core
computing and general purpose GPU processing have started to blur this
distinction however. Phones and tablets these days often feature quadcore
processors and formidable GPUs, generally in a single SoC (System on a Chip)
such as the NVidia Tegra. High-end tablets can play beautiful looking games with
resolutions higher than most computer monitors can even handle, while phones are
able to offer a smoother web browsing experience than low-end laptops can.

This paper will shortly go over the benefits and challenges posed by this
convergence.

\section{Heterogeneous Computing}
Heterogeneous computing, as emphasized by \cite{conf/iccd/KaeliA11} and
\cite{crowley2006impact}, refers to a device using different processing units
with different ISAs.

\bibliographystyle{unsrt}
\bibliography{library}

\end{document}
