\documentclass[a4paper]{article}

\usepackage[english]{babel}
\usepackage{amsmath}
\usepackage{graphicx}

\author{Maarten de Jonge}
\date{\today}
\title{On the convergence of high-performance and embedded computing}

\begin{document}
\maketitle

\begin{abstract}
    Embedded, generally ARM-based, systems are rapidly gaining in performance,
    mainly thanks to the increased heterogeneity and parallelism imposed by the
    observed limit in clock speed. The main challenge remains in making these
    systems programmable in an effective and accessible way. In the future, the
    distinction between embedded ARM systems and traditional x86 desktop and
    server systems will blur.
\end{abstract}

\section{Introduction}
Traditionally, high-performance and embedded computing have been entirely
separate entities; back in the 1990s, phones could do little more than make
calls or send text messages, and nobody would consider running large data
analysis on their microwave. Recent advancements in the fields of multi-core
computing and general purpose GPU processing have started to blur this
distinction however. Phones and tablets these days often feature powerful
quadcore processors and formidable GPUs, generally in a single SoC (System on a
Chip). This paper will shortly go over the cause, the challenges posed by these
developments, and predictions for the future.

\section{Heterogeneous Computing}
Heterogeneous computing, as emphasized by \cite{conf/iccd/KaeliA11} and
\cite{crowley2006impact}, refers to a device using different processing units,
most often CPU, along with a GPU and maybe some specific coprocessors, in a
single platform with high memory bandwidth between each unit. The main factor in
the recent push for homogeneous systems the limit encountered in CPU clock
speed\cite{conf/iccd/KaeliA11}; ever since we arrived at kind of a practical
upper limit for clock speed, manufacturers have been resorting to different ways
of increasing computational performance. It started out with an increasing
number of processor cores per chip, and recently general purpose GPUs have
become a hot topic.  

These paradigms have been especially embraced by embedded systems, because it
allows for high performance with multiple processor cores at relatively low
clock speeds, leading to a far lower power consumption than would be required
for beefed up Pentium 4-era processors with clockspeeds nearing 4ghz.  In the
desktop world, Intel and AMD have both started moving towards further
integration between CPU and GPU with their /emph{Intel Larrabee} and {AMD
Fusion} platforms, while it has been standard in ARM-based consumer electronics
for quite some time (\emph{Nvidia Tegra}, \emph{Qualcomm Snapdragon}, et
cetera).

\section{Challenges}
Although non-CPU devices allow for immense increases in speed in some areas, it
comes at the cost of non-familiarity. For example, programming a GPU
efficiently requires an entirely different approach than programming a CPU at
every layer of abstraction. Especially due to the closed nature of GPU hardware,
programming libraries are limited to the ones specifically supported by the
hardware manufacturers; most notably the open standard
OpenCL\cite{stone2010opencl} and Nvidia's proprietary
CUDA\cite{nvidia2008programming}. Both of these  are extensions of the C
language, although bindings are available for other popular languages such as
Python.

\section{The Future}
Assuming the world doesn't end in 2 days, I predict we will see a further
convergence between embedded systems and traditional desktops. The success of
ARM-based consumer electronics will lead to further development of the ARM
platform, which will eventually become an attractive target for servers due to
ARM's low power consumption compared to x86-based architectures. Adding to this
the fact that Microsoft recently started supporting ARM with their Windows 8
release (and many Linux distributions supporting ARM for many years now), we
might be facing a future where there is little distinction between ARM and x86
platforms, both in terms of performance and the software ecosystems.

\section{Conclusion}
Embedded systems are rapidly approaching traditional desktop computers in terms of
computing power, and both embedded and non-embedded systems are increasing their
heterogeneity. Programming homogeneous systems in an effective and accessible
way is still a challenge, but it seems almost inevitable that the distinctions
between desktop and mobile, and between ARM and x86, will continue fading until
we end with a melting pot of computing architectures. Hopefully, by then there
will be a standard, platform-independent way of programming these, regardless of
the specific homogeneity of the target platform.


\bibliographystyle{plain}
\bibliography{library}

\end{document}
