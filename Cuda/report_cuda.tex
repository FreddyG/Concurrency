\documentclass[a4paper]{article}

\usepackage[english]{babel}
\usepackage{amsmath}
\usepackage{graphicx}

\author{Maarten de Jonge}
\date{\today}
\title{Concurrency and Parallel Programming \\
\large{CUDA assignment}}

\begin{document}
\maketitle

\section*{Wave Equation}
The implementation was done similarly to the previous assignments; each thread
gets a slice of the wave to simulate, although in this case the slices are
markedly smaller than previously due to the enormous amount of threads. The
amount of threads per block has been set at 512, while the number of blocks is
taken as the number of simulated points divided by the number of threads (so
ideally, every thread would only simulate a single point). Because the device
can only run 65535 blocks in parallel, the number of blocks is limited to the
interval $[1, 65535]$.

The time iterations are synchronized by only handling a single iteration in the
kernel, leading to one kernel invocation per time step.

\begin{table}[htbp]
    \centering
    \begin{tabular}{|l|l|l|l|l|}
        \hline
        i\_max    & t\_max   & num\_threads & time pthreads   & time OMP \\
        \hline
        1000     & 1000000 & 1           & 42.4294 & 2.7425  \\
        1000     & 1000000 & 2           & 31.5108 & 42.3953 \\
        1000     & 1000000 & 4           & 80.9137 & 20.493  \\
        1000     & 1000000 & 8           & 167.284 & 23.8034 \\
        1000     & 1000000 & 16          & 350.395 & 21.2966 \\
        10000    & 500000  & 1           & 69.3228 & 12.4777 \\
        10000    & 500000  & 2           & 44.066  & 85.0008 \\
        10000    & 500000  & 4           & 41.1647 & 47.4733 \\
        10000    & 500000  & 8           & 85.3845 & 36.9657 \\
        10000    & 500000  & 16          & 183.268 & 33.3651 \\
        1000000  & 5000    & 1           & 52.108  & 15.6997 \\
        1000000  & 5000    & 2           & 25.6232 & 83.3668 \\
        1000000  & 5000    & 4           & 13.1806 & 42.1536 \\
        1000000  & 5000    & 8           & 7.08639 & 27.3887 \\
        1000000  & 5000    & 16          & 7.06076 & 29.6989 \\
        10000000 & 500     & 1           & 51.4277 & 15.1485 \\
        10000000 & 500     & 2           & 25.8168 & 83.1574 \\
        10000000 & 500     & 4           & 13.0537 & 41.439  \\
        10000000 & 500     & 8           & 7.21982 & 27.4845 \\
        10000000 & 500     & 16          & 7.16489 & 36.9947 \\
        \hline
    \end{tabular}
    \caption{The raw test data, where ``i\_max'' is the number of simulated points
    on the wave and ``t\_max'' is the amount of iterations.}
    \label{table:results}
\end{table}

\begin{table}[htbp]
    \centering
    \begin{tabular}{l|l|l}
        \hline
        i\_max & t\_max & time taken (s) \\
        \hline
        1000     & 1000000 & 4.2354 \\
        10000    & 500000  & 2.3655 \\
        1000000  & 5000    & 1.6818 \\
        10000000 & 500     & 1.9001 \\
        \hline
    \end{tabular}
\end{table}

Table \ref{table:results} shows the results of the old experiments using
pthreads and OpenMP.
For the CUDA experiments, shown in table \ref{table:cuda}, each trial has been
run 10 times and averaged to account for relatively large fluctuations in
runtime.

The results are quite clear; compared to the previous implementations, CUDA is
fast as shit.

\end{document}
