\documentclass[a4paper]{article}

\usepackage[english]{babel}
\usepackage{amsmath}
\usepackage{graphicx}

\author{Maarten de Jonge}
\date{\today}
\title{Concurrency and Parallel Programming \\
\large{CUDA assignment}}

\begin{document}
\maketitle

\section*{Wave Equation}
The implementation was done similarly to the previous assignments; each thread
gets a slice of the wave to simulate, although in this case the slices are
markedly smaller than previously due to the enormous amount of threads. The
amount of threads per block has been set at 512, while the number of blocks is
taken as the number of simulated points divided by the number of threads (so
ideally, every thread would only simulate a single point). Because the device
can only run 65535 blocks in parallel, the number of blocks is limited to the
interval $[1, 65535]$.

The time iterations are synchronized by only handling a single iteration in the
kernel, leading to one kernel invocation per time step.

\begin{table}[htbp]
    \centering
    \begin{tabular}{|l|l|l|l|l|}
        \hline
        i\_max    & t\_max   & num\_threads & time pthreads   & time OMP \\
        \hline
        1000     & 1000000 & 1           & 42.4294 & 2.7425  \\
        1000     & 1000000 & 2           & 31.5108 & 42.3953 \\
        1000     & 1000000 & 4           & 80.9137 & 20.493  \\
        1000     & 1000000 & 8           & 167.284 & 23.8034 \\
        1000     & 1000000 & 16          & 350.395 & 21.2966 \\
        10000    & 500000  & 1           & 69.3228 & 12.4777 \\
        10000    & 500000  & 2           & 44.066  & 85.0008 \\
        10000    & 500000  & 4           & 41.1647 & 47.4733 \\
        10000    & 500000  & 8           & 85.3845 & 36.9657 \\
        10000    & 500000  & 16          & 183.268 & 33.3651 \\
        1000000  & 5000    & 1           & 52.108  & 15.6997 \\
        1000000  & 5000    & 2           & 25.6232 & 83.3668 \\
        1000000  & 5000    & 4           & 13.1806 & 42.1536 \\
        1000000  & 5000    & 8           & 7.08639 & 27.3887 \\
        1000000  & 5000    & 16          & 7.06076 & 29.6989 \\
        10000000 & 500     & 1           & 51.4277 & 15.1485 \\
        10000000 & 500     & 2           & 25.8168 & 83.1574 \\
        10000000 & 500     & 4           & 13.0537 & 41.439  \\
        10000000 & 500     & 8           & 7.21982 & 27.4845 \\
        10000000 & 500     & 16          & 7.16489 & 36.9947 \\
        \hline
    \end{tabular}
    \caption{The raw test data, where ``i\_max'' is the number of simulated points
    on the wave and ``t\_max'' is the amount of iterations.}
    \label{table:results}
\end{table}

\begin{table}[htbp]
    \centering
    \begin{tabular}{l|l|l}
        \hline
        i\_max & t\_max & time taken (s) \\
        \hline
        1000     & 1000000 & 4.2354 \\
        10000    & 500000  & 2.3655 \\
        1000000  & 5000    & 1.6818 \\
        10000000 & 500     & 1.9001 \\
        \hline
    \end{tabular}
    \caption{The CUDA implementation at various parameter values.}
    \label{table:cuda}
\end{table}

Table \ref{table:results} shows the results of the old experiments using
pthreads and OpenMP.
For the CUDA experiments, shown in table \ref{table:cuda}, each trial has been
run 10 times and averaged to account for relatively large fluctuations in
runtime.

The results are quite clear; compared to the previous implementations, CUDA is
really really fast.

\begin{table}[htbp]
    \centering
    \begin{tabular}{l|l|l}
        \hline
        threads per block & time taken (s) \\
        \hline
        8    & 15.0367 \\
        16   & 9.2053 \\
        32   & 5.15237 \\
        64   & 3.27603 \\
        128  & 2.457 \\
        256  & 2.28942 \\
        512  & 2.03195 \\
        1024 & 1.65433 \\
        \hline
    \end{tabular}
    \caption{10000000 wave points over 500 iterations, using various numbers of
    threads per block.}
    \label{table:threads}
\end{table}

Table \ref{table:threads} shows the effect of the number of threads per block,
from 8 up to the maximum of 1024. $i\_max$ and $t\_max$ are kept constant at
10000000 and 500, respectively. These results do not take fluctuations into
account. There is definitely a constant increase in speed with higher numbers of
threads, but there are clear diminishing returns. Because the number of blocks
in 1 dimension is limited to 65535, a low number of threads will equal a larger
chunk of the wave to be handled per thread, and thus less parallelism.

\section{Reduction}
The first version of the reduction algorithm uses a simple architecture with 1
block and 512 threads per block. Each thread reduces a chunk of the input array,
reducing the array to one of 512 elements. Then 1 of threads further reduces
that array to a single value.
Sample output is as follows, using a randomly filled array of 10 million
doubles:
\begin{verbatim}
Parallel max: 999.999711
Time taken: 0.397627 seconds

Sequential max: 999.999711
Time taken: 0.017945 seconds
\end{verbatim}
Although this version is correct, it's slower than doing it sequentially.

\end{document}
